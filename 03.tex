\chapter{Algebraic Stacks}
\begin{center}
	{\huge Speaker: Pietro Leonardini}
\end{center}
\bigskip

\section{Motivation: moduli problems}
\subsection{The dream}
Let us recall what an elliptic curve is:
\begin{definition}[Elliptic curve]
An \textbf{elliptic curve} over $k$ is a 1-dimensional scheme $C\to \Spec k$ over $k$ which is geometrically integral, proper, smooth and of genus\footnote{the genus is $\dim_k H^1(C,\Oc_C)$} 1 with a fixed $k$-rational point $e:\Spec k\to C$.
\end{definition}

Since the relative approach is central in the theory of schemes, we may want to define an elliptic curve over some base scheme $S$. The most sensible approach is to say that an elliptic curve over $S$ should be a ``family of elliptic curves parametrized by $S$". Formally:

\begin{definition}[Elliptic curve over $S$]
An \textbf{elliptic curve} over $S$ is a proper, smooth, flat\footnote{Flatness corresponds in a vague sense to the property that the fibers should ``deform continuously". This notion can be made more precise, for example by looking at Hilbert polynomials among other things, but this vague intuition will be enough for our purposes.} morphism of schemes $p:E\to S$ with a fixed section $e:S\to E$ of $p$ such that for all geometric points $\Spec \Omega\to S$, the fiber is an elliptic curve over $\Omega$:
% https://q.uiver.app/#q=WzAsNCxbMSwwLCJFIl0sWzEsMSwiUyJdLFswLDEsIlxcU3BlYyBcXE9tZWdhIl0sWzAsMCwiRVxcdGltZXNfU1xcU3BlY1xcT21lZ2EgIl0sWzMsMF0sWzMsMl0sWzIsMV0sWzAsMSwicCJdXQ==
\[\begin{tikzcd}
	{E\times_S\Spec\Omega } & E \\
	{\Spec \Omega} & S
	\arrow[from=1-1, to=1-2]
	\arrow[from=1-1, to=2-1]
	\arrow["p", from=1-2, to=2-2]
	\arrow[from=2-1, to=2-2]
\end{tikzcd}\]
\end{definition}

When studying elliptic curves it would be useful to find some space $M_{1,1}$ such that for every family $E\to S$ we have a unique $\phi:S\to M_{1,1}$ such that
% https://q.uiver.app/#q=WzAsNCxbMCwwLCJFIl0sWzAsMSwiUyJdLFsxLDEsIk1fezEsMX0iXSxbMSwwLCJcXEVjIl0sWzAsMV0sWzMsMl0sWzAsM10sWzEsMiwiXFxwaGkiLDJdXQ==
\[\begin{tikzcd}
	E & \Ec \\
	S & {M_{1,1}}
	\arrow[from=1-1, to=1-2]
	\arrow[from=1-1, to=2-1]
	\arrow[from=1-2, to=2-2]
	\arrow["\phi"', from=2-1, to=2-2]
\end{tikzcd}\]
This $\phi$ should be thought of naively as the map that to the point $P\in S(\Omega)$ assigns the point of $M_{1,1}(\Omega)$ that corresponds to the isomorphism class of the elliptic curve given by the fiber of $p:E\to S$ over $P$.

In other words, we would be very happy if all families of elliptic curves could be described as the pullback of some \textit{universal family} $\Ec$ over $M_{1,1}$ along some morphism $S\to M_{1,1}$. Intuitively, the universal family should be the one defined by setting the fiber over a point of $M_{1,1}$ to be the elliptic curve corresponding to that point. 
\bigskip

A useful way to formulate this problem is to define the following functor\footnote{we technically have no guarantee that families over a scheme should form a set. This issue will be solved automatically later when we consider a fibered category over $\Sch$ instead of a functor.}
\[F:\functorDef{\Sch\op}{\Set}{S}{\cpa{E\to S\mid \text{elliptic curves}}/\text{iso.}}{f:T\to S}{(E_1\to S)\mapsto (f^\ast E_1\to T)}\]
We can re-formulate the requirements on $M_{1,1}$ by saying that $M_{1,1}$ should \textbf{represent} this functor, i.e. there is a natural isomorphism of functors
\[h_{M_{1,1}}\cong F\]
where $h_{M_{1,1}}$ is the contravariant $\Hom$-functor associated to $M_{1,1}$.
\medskip

We recall the following simple but foundational result
\begin{theorem}[Yoneda lemma]
There is a natural correspondence
\[\Hom(h_X,F)\leftrightarrow F(X)\]
\end{theorem}
\begin{proof}[Sketch]
Suppose we have a natural transformation $\phi:h_X\to F$, then we can find an element of $F(X)$ by taking $\phi_X(id_X)=\xi$.

Let us now fix an element $\xi\in F(X)$. Let $T$ be a scheme. We may construct a map $\Hom(T,X)\to F(T)$ by sending $f:T\to X$ to $F(f)(\xi)$. If $g:T\to S$ is a morphism then $\eta:f\mapsto g\circ f$ is mapped to $F(g):F(S)\to F(T)$. It is easy to check that this defines a natural transformation $h_X\to F$ and that the two constructions are inverses of each other.
\end{proof}
\begin{corollary}
The functor $\Cc\to \Fun(\Cc\op,\Set)$ that sends $X$ to $h_X$ is fully faithful.
\end{corollary}


\subsection{A good attempt}
Let us try to find some way of parametrizing isomorphism classes of elliptic curves. The main idea comes from the following fact:

\begin{fact}
If $\cha k\neq2,3$ then every elliptic curve is isomorphic over $\ol k$ to some
\[E_\la:y^2=x(x-1)(x-\la)\] 
for some $\la\in k\bs\cpa{0,1}$.
\end{fact}

\begin{definition}[$j$-invariant]
Let
\[j(E_\la)=2^8\frac{(\la^2-\la+1)^3}{\la^2(\la-1)^2}\]
\end{definition}

\begin{fact}
The map
\[\phi:\funcDef{\ol k\bs\cpa{0,1}}{\ol k}{\la}{j(E_\la)}\]
is surjective and $6:1$ except in $j=0$ and $j=1728$, where it is $2:1$ and $3:1$ respectively.
\end{fact}

\begin{fact}
Two elliptic curves are isomorphic over $\ol k$ if and only if they have the same $j$-invariant.
\end{fact}


\begin{theorem}
Let $E\to S$ be an elliptic curve, then there exists a Zariski affine open cover of $S$ given by $\cpa{U_i=\Spec A_i}$ such that for all $i$
% https://q.uiver.app/#q=WzAsNCxbMCwwLCJFX2k6PVxcU3BlY1xccGF7XFxkZnJhY3tBX2lbeCx5XX17KHleMi14XjMtQXgtQil9fSJdLFswLDEsIlxcU3BlYyBBX2kiXSxbMSwxLCJTIl0sWzEsMCwiRSJdLFswLDFdLFszLDJdLFswLDNdLFsxLDJdXQ==
\[\begin{tikzcd}
	{E_i:=\Spec\pa{\dfrac{A_i[x,y]}{(y^2-x^3-Ax-B)}}} & E \\
	{\Spec A_i} & S
	\arrow[from=1-1, to=1-2]
	\arrow[from=1-1, to=2-1]
	\arrow[from=1-2, to=2-2]
	\arrow[from=2-1, to=2-2]
\end{tikzcd}\]
for some $A,B\in A_i$.
\end{theorem}

\begin{remark}
With $E\to S$ elliptic curve and a cover like above, the $j$-invariants of $E_i\to \Spec A_i$ are sections of the structure sheaf of $\Spec A_i$ which coincide on the intersection, so they glue to a section $j\in \Gamma(S,\Oc_S)$.
\end{remark}

\noindent
From this discussion it follows that the best candidate for $M_{1,1}$ is $\A^1$ where the curve over $j\in \A^1$ is the one with that $j$-invariant.
Unfortunately $\A^1$ does not work.

\begin{example}
Suppose $\A^1$ represents the functor $F$ from before. Consider then the two families
% https://q.uiver.app/#q=WzAsNCxbMCwwLCJcXFNwZWNcXHBhe1xcZnJhY3trW3gseSx0XlxccG1dfXsoeV4yLXheMyt0KX19Il0sWzAsMSwiXFxTcGVjIGtbdCx0XFxpaV0iXSxbMSwxLCJcXEFeMSJdLFsxLDAsIlxceGkiXSxbMCwxXSxbMywyXSxbMCwzXSxbMSwyXV0=
\[\begin{tikzcd}
	{\Spec\pa{\frac{k[x,y,t^\pm]}{(y^2-x^3+t)}}} & \xi \\
	{\Spec k[t,t\ii]} & {\A^1}
	\arrow[from=1-1, to=1-2]
	\arrow[from=1-1, to=2-1]
	\arrow[from=1-2, to=2-2]
	\arrow[from=2-1, to=2-2]
\end{tikzcd}\]
and
% https://q.uiver.app/#q=WzAsNSxbMCwxLCJcXFNwZWNcXHBhe1xcZnJhY3trW3gseSx0XlxccG1dfXsoeV4yLXheMysxKX19Il0sWzAsMiwiXFxTcGVjIGtbdCx0XFxpaV0iXSxbMSwyLCJcXEFeMSJdLFsxLDEsIlxceGkiXSxbMCwwLCJcXFNwZWNcXHBhe1xcZnJhY3trW3gseV19eyh5XjIteF4zKzEpfX1cXHRpbWVzXFxTcGVjXFxwYXtrW3Ree1xccG0xfV19Il0sWzAsMV0sWzMsMl0sWzAsM10sWzEsMl0sWzQsMCwiPSIsMSx7InN0eWxlIjp7ImJvZHkiOnsibmFtZSI6Im5vbmUifSwiaGVhZCI6eyJuYW1lIjoibm9uZSJ9fX1dXQ==
\[\begin{tikzcd}
	{\Spec\pa{\frac{k[x,y]}{(y^2-x^3+1)}}\times\Spec\pa{k[t^{\pm1}]}} \\
	{\Spec\pa{\frac{k[x,y,t^\pm]}{(y^2-x^3+1)}}} & \xi \\
	{\Spec k[t,t\ii]} & {\A^1}
	\arrow["{=}"{description}, draw=none, from=1-1, to=2-1]
	\arrow[from=2-1, to=2-2]
	\arrow[from=2-1, to=3-1]
	\arrow[from=2-2, to=3-2]
	\arrow[from=3-1, to=3-2]
\end{tikzcd}\]
If $\A^1$ represented the functor, these two families should be isomorphic since for every specific choice $t\neq 0$ we have that $y^2=x^3-t$ and $y^2=x^3-1$ have the same $j$-invariant and so both families induce the same (constant) map $\Spec k[t^{\pm1}]\to \A^1$.

However, the two families are NOT isomorphic by the following argument: It is possible to prove via the $j$-invariants that two Weierstrass equations yield isomorphic elliptic curves over $\Spec k[t^{\pm1}]$ if and only if there exists some $u\in k[t^{\pm1}]$ such that substituting $x$ with $u^2x$ and $y$ with $u^3y$ transforms one equation into the other up to factoring out powers of $u$. Therefore we require for there to exist some $u$ such that
\[u^6y^2-u^6x^3+t=u^k(y^2-x^3+1)\coimplies t=u^6\]
and this is impossible.
\end{example}

\begin{remark}
A slogan to keep in mind is
\begin{center}
    \textit{The presence of non-trivial automorphisms prevent the moduli problem from having a fine moduli space.}
\end{center}
\end{remark}

\begin{example}
If we consider the same families from before but now over the base $\Spec\frac{k[t^{\pm1},w]}{(t-w^6)}$ instead of $\Spec k[t^{\pm 1}]$, they now become isomorphic, that is
\[\frac{k[x,y,t^\pm,w]}{(y^2-x^3+1,w^6-t)}\cong \frac{k[x,y,t^\pm,w]}{(y^2-x^3+t,w^6-t)}.\]
This is because now we can choose $u=w$ with the notation from before. 

The idea behind this observation is that if we allow ourselves to look at finer topologies\footnote{In this case, the natural map $\Spec\frac{k[t^{\pm1},w]}{(t-w^6)}\to\Spec k[t^{\pm 1}]$ is an \'etale cover.} on $\Sch$ we may be able to solve our representability issue. We are starting to see how viewing $F$ as a functor is not enough; we should view it as some kind of sheaf.
\end{example}


\subsection{Our dreams crumble?}

So, what now? We have two options:
\begin{itemize}
    \item Rigidify the problem hoping that we get something representable
    \item Enlarge the category among which we search for the moduli space (Stacky approach) 
\end{itemize}

\begin{notation}
From now on we identify the scheme $X$ with the functor $h_X:\Sch\op\to \Set$ and we identify it with the appropriate fibered category over $\Sch$. 
\end{notation}
\begin{remark}
$X(k)=\Hom(\Spec k,X)$.
\end{remark}

\begin{center}
	{\Large The main idea:}
\end{center}
Instead of the functor $F$ from before \textbf{we consider the category} $\Mc_{1,1}$ \textbf{fibered in groupoids} over $\Sch$ defined the same way.

\begin{remark}
The issue with families forming sets is now irrelevant. To be more precise, the groupoid associated to a scheme $S$ is the one whose objects are elliptic curves over $S$ and whose morphisms are isomorphisms of families (making this category a groupoid by construction).
\end{remark}


\begin{theorem}
The fibered category $\Mc_{1,1}$ is a stack for the fpqc topology\footnote{and thus for the coarser topologies like Zariski or \'Etale too.}.
\end{theorem}
\begin{proof}
Olsson \cite{olsson2016algebraic}, chapter 13
\end{proof}

\begin{remark}
A morphism $\phi :X\to\Mc_{1,1}$ of stacks corresponds by (a version of) the Yoneda lemma to an element of the fiber $\Mc_{1,1}(X)$.
\bigskip

\noindent Said another way, there is a correspondence between families of curves over $X$ (the category $\Mc_{1,1}(X)$) and morphisms $X\to \Mc_{1,1}$.
\end{remark}

From a categorical point of view, $\Mc_{1,1}$ does exactly what we want. Now we try to understand in which ways it can be thought of as a space.


\section{Algebraic stacks}
If we want to think about stacks in a geometric way, we may want to impose some ``good behaviour" conditions. For example, one thing we may require is that if two ``normal" objects (for example schemes) live inside our stack, their intersection should still be a ``normal" object. The key definition to formalize this idea is
\begin{definition}[Representable morphism]
Let $\Xc\to \Yc$ be a morphism of stacks, then it is said to be \textbf{representable} if for all schemes $T$ and morphisms $T\to \Yc$ there exists an algebraic space\footnote{An \textbf{algebraic space} is a gluing of affine schemes with respect to the \'etale topology instead of the Zariski topology.} $V$ such that
% https://q.uiver.app/#q=WzAsNCxbMCwwLCJWIl0sWzAsMSwiXFxYYyJdLFsxLDEsIlxcWWMiXSxbMSwwLCJUIl0sWzAsMV0sWzMsMl0sWzAsM10sWzEsMl1d
\[\begin{tikzcd}
	V & T \\
	\Xc & \Yc
	\arrow[from=1-1, to=1-2]
	\arrow[from=1-1, to=2-1]
	\arrow[from=1-2, to=2-2]
	\arrow[from=2-1, to=2-2]
\end{tikzcd}\]
is cartesian.

We say that a morphism of stacks $\Xc\to \Yc$ has a property $\Pc$ for $\Pc$ a property of morphisms of schemes/algebraic spaces which is stable under base change if the morphism is representable and for all $T\to\Yc$, the induced arrow $V\to T$ as above has the property $\Pc$.
\end{definition}

\begin{remark}
The properties ``begin smooth", ``begin surjective" and ``begin \'etale" are stable under base change, so it makes sense to talk about smooth/surjective/\'etale morphisms of stacks.
\end{remark}

\begin{definition}[Algebraic stack]
A stack over a scheme $S$, i.e. a morphism of stacks $\Xc\to S$, is said to be \textbf{algebraic} if
\begin{enumerate}
\item the diagonal $\Delta:\Xc\to \Xc\times_S\Xc$ is representable and 
\item there exists a smooth and surjective morphism $\pi:X\to \Xc$ for $X$ scheme. This scheme is called an \textbf{atlas} for the algebraic stack.
\end{enumerate}
If $\pi:X\to \Xc$ is \'etale then $\Xc\to S$ is called a \textbf{Deligne-Mumford stack}.
\end{definition}

\begin{remark}
Property 1. looks very mysterious but the following is an equivalent and, perhaps clearer, formulation: for all $T,U$ schemes, the fibered product $T\times_\Xc U$ exists and is an algebraic space. We may think of this as saying that ``intersecting schemes in a stack gives an algebraic space".

Thinking about this set-theoretically may help in understanding the equivalence.
\end{remark}

\begin{remark}
Property 2. can be thought of as a ``finiteness" condition. The case of Deligne-Mumford stacks makes this even clearer because \'etale morphisms can be thought of as unramified covers, so we are saying that our stack admits a cover with total space being a scheme.
\end{remark}






\section{Quotient stacks}

In this section we will explore quotient stacks and they essentially allow us to generalize in algebraic geometry the concept of taking the quotient of a space under the action of a group. There is a different (but related) machinery for the same general goal called \textit{Geometric Invariant Theory} (GIT) but the aspired results are different. In GIT one is concerned with \textit{when quotients exist} in some category (usually schemes), while quotient stacks allow us to \textit{define something that behaves like a quotient} without worrying about these existence questions but forcing us to generalize our notion of ``space".


Like any stack, quotient stacks are categories. Let us define the objects:

\begin{definition}[$G$-fibration]
Let $X$ be a scheme, $G$ a group scheme. A \textbf{$G$-fibration} over $X$ is a scheme $\xi$ with an action $\mu:G\times \xi\to \xi$ and a $G$-invariant morphism $\pi:\xi\to X$, that is, we have a commutative diagram
% https://q.uiver.app/#q=WzAsNCxbMCwwLCJHXFx0aW1lcyBcXHhpIl0sWzAsMSwiXFx4aSJdLFsxLDAsIlxceGkiXSxbMSwxLCJYIl0sWzAsMiwiXFxtdSJdLFswLDEsIlxccGlfMiIsMl0sWzIsMywiXFxwaSJdLFsxLDMsIlxccGkiLDJdXQ==
\[\begin{tikzcd}
	{G\times \xi} & \xi \\
	\xi & X
	\arrow["\mu", from=1-1, to=1-2]
	\arrow["{\pi_2}"', from=1-1, to=2-1]
	\arrow["\pi", from=1-2, to=2-2]
	\arrow["\pi"', from=2-1, to=2-2]
\end{tikzcd}\]
\end{definition}

\begin{example}
$\xi=G\times X$ with the obvious projection $\xi\to X$ is a $G$-fibration. This is called the \textbf{trivial} $G$-fibration.
\end{example}

\begin{definition}[Principal $G$-bundle]
A \textbf{principal $G$-bundle} with respect to the fpqc (or fppf, \'etale etc) topology is a $G$-fibration which is locally trivial, i.e. there exists a cover $\cpa{U_i\to X}$ such that $\xi\res {U_i}\to U_i$ is isomorphic to $G\times U_i\to U_i$.
\end{definition}




\begin{proposition}
Let $X$ be a scheme and $G$ a linear algebraic group acting on $X$ via $\mu:X\times G\to X$. Assume that the action is free\footnote{for all schemes $\mu_T:X(T)\times G(T)\to X(T)$ is a free action of $G(T)$}, then $X/G$ exists as an algebraic space.

Moreover $\rho:X\to X/G$ is a principal $G$-bundle with respect to the \'etale topology and the following is cartesian
% https://q.uiver.app/#q=WzAsNCxbMCwwLCJYXFx0aW1lcyBHIl0sWzAsMSwiWCJdLFsxLDEsIlgvRyJdLFsxLDAsIlgiXSxbMCwxLCJcXG11IiwyXSxbMywyLCJcXHJobyJdLFswLDMsIlxccGlfMSJdLFsxLDJdXQ==
\[\begin{tikzcd}
	{X\times G} & X \\
	X & {X/G}
	\arrow["{\pi_1}", from=1-1, to=1-2]
	\arrow["\mu"', from=1-1, to=2-1]
	\arrow["\rho", from=1-2, to=2-2]
	\arrow[from=2-1, to=2-2]
\end{tikzcd}\]
\end{proposition}

\begin{remark}\label{ReHowToGetData}
If we fix $f:U\to X/G$ and take $\xi$ to be the pullback of $X\to X/G$ along $U\to X/G$ 
% https://q.uiver.app/#q=WzAsNCxbMCwwLCJcXHhpIl0sWzAsMSwiVSJdLFsxLDAsIlgiXSxbMSwxLCJYL0ciXSxbMCwyLCJcXGFsIl0sWzAsMSwiXFxwaSIsMl0sWzIsMywiXFxyaG8iXSxbMSwzLCJmIiwyXV0=
\[\begin{tikzcd}
	\xi & X \\
	U & {X/G}
	\arrow["\al", from=1-1, to=1-2]
	\arrow["\pi"', from=1-1, to=2-1]
	\arrow["\rho", from=1-2, to=2-2]
	\arrow["f"', from=2-1, to=2-2]
\end{tikzcd}\]
then
\begin{enumerate}
    \item $\pi:\xi\to U$ is a $G$-bundle and
    \item $\al:\xi\to X$ is $G$-equivariant.
\end{enumerate}
\end{remark}


\begin{definition}[]
Let $X$ be a scheme and $G$ a smooth linear algebraic group acting on $X$, the \textbf{quotient stack} $[X/G]$ is the fibered category over schemes
\[U\mapsto \cpa{(\pi:\xi\to U, \al:\xi\to X)\mid \pi\text{ principal $G$-bundle, $\al$ $G$-equivariant}}/\sim\]
If $X=\Spec k$, $[\Spec k/G]=\B G$ is called the \textbf{classifying space} of $G$.
\end{definition}

Let's pause for a moment to understand why this definition makes sense. As defined, the quotient stack is the moduli space for the problem of classifying principal $G$-bundles with an equivariant map to $X$. So maps to $[X/G]$ correspond to this data.

On the other hand, what should maps to a quotient of $X$ by $G$ entail? As in remark (\ref{ReHowToGetData}) by pullback of $X\to X/G$ via such a map we should get a $G$-bundle and a $G$-equivariant map to $X$.

Putting these facts together, $[X/G]$ behaves like a categorical quotient should.


\begin{theorem}
    $[X/G]$ is an algebraic stack
\end{theorem}
\begin{proof}[Partial idea]
Let us convince ourselves that $[X/G]$ should be an algebraic stack with atlas $X$:
\setlength{\leftmargini}{0cm}
\begin{itemize}
\item[] We want a morphism $\pi:X\to [X/G]$ (which intuituvely should be the quotient map). Such a morphism corresponds via the Yoneda lemma to a family of principal $G$-bundles over $X$, which we can choose to be
% https://q.uiver.app/#q=WzAsMyxbMCwwLCJYXFx0aW1lcyBHIl0sWzAsMSwiWCJdLFsxLDAsIlgiXSxbMCwxLCJcXG11IiwyXSxbMCwyLCJcXHBpXzEiXV0=
\[\begin{tikzcd}
	{X\times G} & X \\
	X
	\arrow["{\pi_1}", from=1-1, to=1-2]
	\arrow["\mu"', from=1-1, to=2-1]
\end{tikzcd}\]
\item[] If we fix $\vp:U\to [X/G]$, this corresponds to a family
% https://q.uiver.app/#q=WzAsMyxbMCwwLCJcXHhpIl0sWzAsMSwiVSJdLFsxLDAsIlgiXSxbMCwxLCJcXGV0YSIsMl0sWzAsMiwiXFxhbCJdXQ==
\[\begin{tikzcd}
	\xi & X \\
	U
	\arrow["\al", from=1-1, to=1-2]
	\arrow["\eta"', from=1-1, to=2-1]
\end{tikzcd}\]
It turns out that $\xi=X\times_{[X/G]}U$ so we would need to show that $\eta:\xi\to U$ is smooth and surjective (not obvious but provable).
\end{itemize}
\end{proof}






\subsection{Back to elliptic curves}
Let us consider the scheme\footnote{$U\subseteq\A^2$ is the possible pairs $(A,B)$ of coefficients in a Weierstrass form for an elliptic curve. Localizing at the discriminant enforces that for any point $(A,B)\in U$, the curve $y^2=x^3+Ax+B$ is an elliptic curve.} $U=\Spec k[A,B,1/\Delta]$ where $\Delta=-16(4A^3+27B^2)$, and let $G=\G_m=k[u^{\pm 1}]$.
$G$ actis on $U$ by looking at equivalent Weierstrass forms via the change of variables
\[x\mapsto u^2 x, y\mapsto u^3 y.\]
Concretely, the action maps $A$ to $u^{-4}A$, $B$ to $u^{-6}B$ and $\Delta$ to $u^{-12}\Delta$.

\begin{fact}
$\Mc_{1,1}\cong [U/G]$.
\end{fact}







