\chapter{More functoriality and vector bundles}
\begin{center}
	{\huge Speaker: Francesco Minnocci}
\end{center}
\bigskip

\noindent
The main references for this chapter are \cite{fulton1984intersection} and \cite{eisenbud20163264}. 



\section{Pullback}

Let $f:X\to Y$ be a morphism. In the last chapter we managed to construct a pushforward when $f$ is proper. Our first goal in this chapter is to construct a pullback, at least in some cases.


To be more concrete, we want a map $Z(Y)\to Z(X)$ which preserves rational equivalence (so it induces a map at the level of Chow groups) and the intersection product.

\subsection{Divisors and rational equivalence}
\begin{definition}[]
Let $V\subseteq X$ be a subvariety with generic point $\xi$ and fix $f\in k(X)^\ast$. Note that $f=f_1/f_2$ with $f_1,f_2\in \Oc_{X,\xi}$\footnote{this amounts to saying that, locally, a rational function is the ratio of regular functions.}. We define the \textbf{order} of $f$ along $V$ to be
\[\ord_Vf:=\length_{\Oc_{X,\xi}}(\Oc_{X,\xi}/(f_1))-\length_{\Oc_{X,\xi}}(\Oc_{X,\xi}/(f_2)).\]
\end{definition}


\begin{definition}[]
Let $f\in k(X)^\ast$, we define its \textbf{associated divisor} to be
\[\mathrm{div} f=\sum_V\ord_V(f)[V].\]
\end{definition}


\begin{proposition}[]
A cycle $\al\in Z^k(X)$ is rationally equivalent to $0$ if and only if there exists a finite number of varieties $W_i$ together with inclusions\footnote{recall that closed immersions are proper morphisms.} $\iota_i:W_i\to X$ and rational functions $f_i\in k(W_i)^\ast$ such that $\al=\sum (\iota_i)_\ast[\mathrm{div}(f_i)]$.
\end{proposition}

\subsection{Regular immersions}
\begin{definition}[]
A \textbf{regular immersion} of codimension $r$ is a locally closed immersion given locally by a quotient of an ideal which is generated by a regular sequence of length $r$.
\end{definition}

\begin{notation}
In this chapter, when we say flat we also mean that all fibers should the same dimension. This is automatic when the codomain is connected because of the properties of flat morphisms.
\end{notation}

\begin{fact}[]
If $Y$ is smooth then we can factor any morphism $f:X\to Y$ as
% https://q.uiver.app/#q=WzAsMyxbMCwwLCJYIl0sWzEsMCwiWSJdLFsxLDEsIlhcXHRpbWVzIFkiXSxbMCwxLCJmIl0sWzIsMSwiXFxwcl8yIiwyXSxbMCwyLCJcXEdhbW1hX2YiLDJdXQ==
\[\begin{tikzcd}
	X & Y \\
	& {X\times Y}
	\arrow["f", from=1-1, to=1-2]
	\arrow["{\Gamma_f}"', from=1-1, to=2-2]
	\arrow["{\pr_2}"', from=2-2, to=1-2]
\end{tikzcd}\]
where $\Gamma_f$ is the graph-morphism, which is a regular immersion, and $\pr_2$ is the projection on the second factor, which is flat. 
\end{fact}

This shows that, for $Y$ smooth, we may construct pullbacks by only considering morphisms which are regular immersions or flat.

\begin{theorem}[]
If $i$ is a regular immersion then $i^\ast$ exists and resprects the functorial properties 
\end{theorem}

\begin{remark}
If pullbacks exist for $X$ smooth, and $\Delta:X\to X\times X$ is a regular immersion then, if $V,W\subseteq X$ are subvarieties, we want
\[[V][W]=\Delta^\ast([V\times W]),\]
that is, we want the product $[V][W]$ to correspond to ``the points of $X$ which under the diagonal map to pairs in $V\times W$", i.e. the intersection.
\end{remark}

\subsection{Flat pullback}
\begin{theorem}[]
If $f:Y\to X$ is flat of codimension $n$ and $V\subseteq X$ is a subvariety, the function
\[f^\ast([V])=[f\ii(V)]\]
is a homomorphism $Z(Y)\to Z(X)$ of degree $n$ (if we grade cycles by dimension).
\end{theorem}



\begin{proposition}[]
Suppose we have a cartesian square
% https://q.uiver.app/#q=WzAsNCxbMSwwLCJYIl0sWzEsMSwiWSJdLFswLDEsIlknIl0sWzAsMCwiWCciXSxbMCwxLCJmIl0sWzMsMiwiZiciLDJdLFsyLDEsImciLDJdLFszLDAsImcnIl1d
\[\begin{tikzcd}
	{X'} & X \\
	{Y'} & Y
	\arrow["{g'}", from=1-1, to=1-2]
	\arrow["{f'}"', from=1-1, to=2-1]
	\arrow["f", from=1-2, to=2-2]
	\arrow["g"', from=2-1, to=2-2]
\end{tikzcd}\]
with $f$ proper and $g$ flat, then for all $\al\in Z(X)$
\[f_\ast'g'^\ast(\al)=g^\ast f_\ast(\al).\]
\end{proposition}


\begin{proposition}[]
Let $f:X\to Y$ be a flat morphism and $\al\in Z^k(Y)$ be an element of $\Rat(Y)$, then $f^\ast\al\sim 0$.
\end{proposition}
\begin{proof}
Without loss of generality, we may assume $\al=[\Phi(0)]-[\Phi(\infty)]$ where $\Phi(t)=\Phi\cap \cpa{t}\times Y$ for $\Phi\subseteq \Pj^1\times Y$ closed and $t\in \Pj^1(k)$.
Consider the base change diagram
% https://q.uiver.app/#q=WzAsNyxbMCwzLCJYIl0sWzEsMywiWSJdLFsxLDIsIlxcUGpeMVxcdGltZXMgWSJdLFswLDIsIlxcUGpeMVxcdGltZXMgWCJdLFsyLDAsIlxcUGpeMSJdLFswLDEsIlciXSxbMSwxLCJcXFBoaSJdLFswLDEsImYiXSxbMywyLCIoaWQsZikiXSxbMiwxLCJxIl0sWzMsMCwicCIsMl0sWzIsNCwiZyJdLFs1LDZdLFs2LDIsIiIsMCx7InN0eWxlIjp7InRhaWwiOnsibmFtZSI6Imhvb2siLCJzaWRlIjoidG9wIn19fV0sWzUsMywiIiwwLHsic3R5bGUiOnsidGFpbCI6eyJuYW1lIjoiaG9vayIsInNpZGUiOiJ0b3AifX19XSxbNSw0LCJoIl1d
\[\begin{tikzcd}
	&& {\Pj^1} \\
	W & \Phi \\
	{\Pj^1\times X} & {\Pj^1\times Y} \\
	X & Y
	\arrow["h", from=2-1, to=1-3]
	\arrow[from=2-1, to=2-2]
	\arrow[hook, from=2-1, to=3-1]
	\arrow[hook, from=2-2, to=3-2]
	\arrow["{(id,f)}", from=3-1, to=3-2]
	\arrow["p"', from=3-1, to=4-1]
	\arrow["g", from=3-2, to=1-3]
	\arrow["q", from=3-2, to=4-2]
	\arrow["f", from=4-1, to=4-2]
\end{tikzcd}\]
where $W=((id, f))\ii(\Phi)$ and $p,q,g,h$ are the appropriate projections.
\begin{align*}
f^\ast(\al)=&f^\ast([\Phi(0)]-[\Phi(\infty)])=f^\ast q_\ast([g\ii(0)]-[g\ii(\infty)])=\\
=&p_\ast((id, f))^\ast([g\ii(0)]-[g\ii(\infty)])=\\
=&p_\ast([h\ii(0)]-[h\ii(\infty)]).
\end{align*}
If $W=\bigcup W_i$ with $W_i$ irreducible, we may write $[W]=\sum m_i[W_i]$ for appropriate $m_i\in\Z$. Note that if $h_i=h\res{W_i}$ then
\[[h\ii(t)]=\sum m_i h_i\ii(t).\]
Since
\[[h_i\ii(0)]-[h_i\ii(\infty)]=[\mathrm{div}(h_i)]\]
this shows that
\[f^\ast(\al)=\sum m_i p_\ast[\mathrm{div}(h_i)]\sim 0.\]
\end{proof}


\begin{corollary}[]
We have a well defined degree-preserving group homomorphism $f^\ast:CH^\bullet(Y)\to CH^\bullet(X)$.
\end{corollary}

\begin{remark}
$f^\ast$ is also a ring homomorphism.
\end{remark}


\begin{remark}
The pullback $f^\ast:CH(Y)\to CH(X)$ induces a $CH(Y)$-module structure on $CH(X)$.
\end{remark}



\begin{proposition}[Projection formula]
If $f$ is a flat and proper morphism between smooth varieties then $f_\ast$ is a homomorphism of $CH(Y)$-modules, i.e.
\[f_\ast(f^\ast(\beta)\cdot \al)=\beta\cdot f_\ast(\al)\]
\end{proposition}


\begin{remark}
Plugging in $\al=1_{CH(X)}=[X]$ we have that $f_\ast f^\ast(\beta)=(\deg f)\beta$.
\end{remark}















\section{Chern classes}
We now move on to one of the first applications of the Chow ring: giving vector bundles their associated Chern classes.

\subsection{Brief review of vector bundles}
\begin{definition}[]
$\Ec$ is a \textbf{vector bundle} if it is a locally free sheaf of $\Oc_X$-modules of some finite rank $r$.
\end{definition}

\begin{remark}
Vector bundles of rank 1 are called \textbf{line bundles}.
\end{remark}

\begin{definition}[]
The \textbf{dual}\footnote{also called \textbf{inverse} in the case of line bundles.} of a vector bundle $\Ec$ is $\Ec^\vee=\Hom(\Ec,\Oc_X)$.
\end{definition}

\begin{definition}[]
An \textbf{algebraic vector bundle} is a morphism $p:E\to X$ of schemes such that for all $x\in X$ there exists $U$ neighborhood of $x$ such that $f\ii(U)\cong \A^r\times U$ and the transition maps are linear fiber-wise.
\end{definition}

\begin{remark}
We have a correspondence between the two notions given by sending a locally free sheaf $\Ec$ to 
\[\ul{\Spec}_{\Oc_X}(\ul{\Sym}_{\Oc_X}(\Ec^\vee))\]
and sending an algebraic vector bundle to its sheaf of local sections.

We can also construct a projective version of bundles by taking
\[\Pj(\Ec)=\ul{\Proj}_{\Oc_X}(\ul{\Sym}_{\Oc_X}(\Ec^\vee)).\]
\end{remark}




\subsection{First Chern class of a line bundle}

Recall that we are taking $X$ to be smooth and noetherian, so in particular also irreducible and has a generic point $\xi$. It follows that non-empty open subsets of $X$ are dense.


\begin{definition}[]
Let $L$ be a line bundle on $X$, then a \textbf{rational section} of $L$ is a germ $f\in L_\xi$.
\end{definition}


\begin{remark}
Since $L$ is locally free of rank $1$, $L_\xi\cong \Oc_{X,\xi}=k(X)$ as $\Oc_{X,\xi}$-modules. It follows that a nonzero rational section $\sigma$ always exists because $k(X)^\ast\neq \emptyset$. 
\end{remark}

\begin{proposition}[]
A nonzero rational section determines a divisor on $X$. Moreover, all divisors obtained this way are rationally equivalent.
\end{proposition}
\begin{proof}
Let $\cpa{U_i}$ be a trivializing open cover for $L$ and let us fix isomorphisms $\vp_i:L_{U_i}\cong \Oc_{U_i}$. Set $f_i=(\vp_i)_\xi(\sigma)$. This allows us to construct divisors $\mathrm{div}(f_i)$ on each $U_i$. On the intersections $U_{ij}$ both of the divisors we got separately on $U_i$ and $U_j$ agree because the automorphism of $\Oc(U_{ij})$ induced by the equality $L_{U_{i}}\res{U_{ij}}=L_{U_j}\res{U_{ij}}$ must be multiplication by an invertible element of $g_{ij}\in \Oc(U_{ij})^\ast$ (that is, $f_j\res{U_{ij}}=g_{ij}f_j\res{U_ij}$) and therefore
\[\mathrm{div}(f_j)\res{U_{ij}}=\cancelto{0}{\mathrm{div}(g_{ij})}+\mathrm{div}(f_i)\res{U_{ij}}.\]
Changing the isomorphisms $L_{U_i}\cong \Oc_{U_i}$ does not change the associated divisor for similar reasons.

Let us call this divisor $\mathrm{div}(\sigma)$.
Since $L_\xi\cong k(X)$, it follows that two different nonzero rational sections $\sigma$ and $\tau$ must determine a rational function $f=\sigma/\tau\in k(X)^\ast$. It follows that
\[\mathrm{div}(\sigma)=\mathrm{div}(f)+\mathrm{div}(\tau),\]
that is, $\mathrm{div}(\sigma)\sim \mathrm{div}(\tau)$.
\end{proof}

\begin{definition}[]
We define $c_1(L)$ to be the class in $CH^1(X)$ of any divisor associated to a nonzero rational section of $L$.
\end{definition}

\begin{example}
Let $X=\Pj^n$, $L=\Oc(d)$ with $d\geq 0$. Then $c_1(L)=[H]=d[h]$ where $h$ is a hyperplane (because $H$ is a degree $d$ hypersurface). This comes from the fact that global sections of $\Oc(d)$ are homogeneous polynomials of degree $d$.
\end{example}

\begin{proposition}[]
If $X$ is smooth, $c_1:\Pic(X)\to CH^1(X)$ is an isomorphism. 

In particular, if $\sigma,\ \sigma'$ are rational sections of $L$ and $L'$ respectively, the rational section $\sigma\otimes \sigma'$ of $L\otimes L'$ behaves as follows: 
\[\mathrm{div}(f\otimes f')=\mathrm{div}(f)+\mathrm{div(f')}.\]
\end{proposition}


\subsection{Axiomatic definition of Chern classes}
There is another definition of $c_1$ which works for all vector bundles (and yields as many classes as the rank of the bundle itself). 

\begin{theorem}[]
If $\Ec$ is a rank $r$ vector bundle on $X$ then
$CH^\bullet(\Pj(\Ec))$ is a free $CH^\bullet(X)$-module of rank $r$. 

Moreover, if $L=\Oc_{\Pj(\Ec)}(1)$ and $h=c_1(L)\in CH^1(\Pj(\Ec))$, then $\cpa{1,\cdots, h^{r-1}}$ is a basis of $CH^\bullet(\Pj(\Ec))$ and the only relation is of the form
\[h^r+c_1(\Ec)h^{r-1}+\cdots+c_r(\Ec)=0\]
for some unique $c_i(\Ec)\in CH^i(X)$.
\end{theorem}

\begin{remark}
This is a generalization of the fact that $CH(\Pj^n)\cong \Z[x]/(x^{n+1})$ for $x$ hyperplane.
\end{remark}


\begin{example}
Let $\Ec=L$ be a line bundle and $\pi:\Pj(L)\to X$ be its projectivization. Note that\footnote{$\pi^\ast L$ is the tautological bundle of $\Pj(\Ec)$} $\Oc_{\Pj(L)}(1)=\pi^\ast L^\vee$. 
Let $h=c_1(\pi^\ast L^\vee)=-\pi^\ast c_1(L)=-c_1(L)[\Pj(L)]$ be the generator of the Chow group. With the new definition of $c_1$ we got in the previous theorem (which we now denote $\wt c_1$ to avoid confusion) we have
\[h^1+\wt c_1(L)h^0=0\implies h=-\wt c_1(L)[\Pj(L)]\]
so indeed the two definitions of $c_1$ coincide for line bundles.
\end{example}



\begin{definition}[]
We define the \textbf{total Chern class} of $\Ec$ to be
\[c(\Ec)=\sum_{i=0}^rc_i(\Ec)\]
\end{definition}

\begin{proposition}[]
The following hold
\begin{itemize}
\item If $f$ is flat, $f^\ast(c(\Ec))=c(f^\ast(\Ec))$
\item If we have an exact sequence $0\to \Ec\to \Fc\to \Gc\to 0$ of vector bundles then
\[c(\Fc)=c(\Ec)c(\Gc),\]
i.e.
\[c_k(\Fc)=\sum_{i+j=k}c_i(\Ec)c_j(\Gc)\]
Moreover, in the case of a split exact sequence we have $c(\Ec\oplus \Gc)=c(\Ec)c(\Gc)$.
\end{itemize}
\end{proposition}

\subsection{Splitting principle}

\begin{theorem}[Splitting principle]
If $E\to X$ vector bundle of rank $n$ then there exists $\vp:Y\to X$ smooth such that 
\begin{enumerate}
    \item $\vp^\ast:CH(Y)\to CH(X)$ is injective
    \item $\vp^\ast E\to Y$ has a filtration
    \[0=E_0\subsetneq E_1\subsetneq\cdots, \subseteq E_n=\vp^\ast E\]
    with $E_i/E_{i+1}$ locally free of rank $1$.
\end{enumerate}	
\end{theorem}


\begin{corollary}[]
We have that
\[\vp^\ast c(E)=c(\vp^\ast E)=\prod_{i=0}^{n-1} c(E_i/E_{i+1})\]
\end{corollary}

\begin{example}
If $E=\bigoplus_{i=1}^r L_i$ and $\al_i=c_1(L_i)$ then
\[c(E)=\prod_{i=1}^r(1+\al_i),\quad c(E^\vee)=\prod_{i=1}^r(1-\al_i),\quad c_k(E^\vee)=(-1)^kc_k(E)\]
\end{example}


\begin{definition}[]
If $\Ec$ is a vector bundle of rank $r$, $\det \Ec=\bigwedge^r\Ec$.
\end{definition}

\begin{proposition}[]
$c_1(E)=c_1(\det E)$.
\end{proposition}
\begin{proof}[Idea.]
If $E=\bigoplus L_i$ then $c_1(E)=\sum c_1(L_i)=c_1(\bigotimes L_i)=c_1(\det(\bigotimes L_i))$. The general case follows from this combined with the splitting principle.
\end{proof}














