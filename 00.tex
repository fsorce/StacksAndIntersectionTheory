\chapter*{Foreword}

These notes contain the topics that were discussed during the reading group
\begin{center}
    \textit{Intersection Theory on Moduli Spaces}
\end{center}
which met in Pisa during the second semester of the 2024/25 accademic year.
The reading group was organized by Leonardini Pietro and Tarini Bernardo. 

Each chapter corresponds to a lecture. The name of the speaker is shown at the start of the appropriate chapter.


Some information was added while typing the notes to make the delivery clearer, but these notes have no aspiration to be thorough. For further reading you may refer to the bibliography. Each chapter also mentions which sources are more relevant.


\medskip


The goal of the reading group was to lay out the basic theory of algebraic stacks and intersection theory in order to combine the two in the context of moduli theory.

In the first two chapters we lay out the basic categorical notions involved and we define stacks, which are then upgraded to algebraic stacks in the third chapter with the motivation of moduli theory.

The fourth, fith and sixth chapters give a brief overview of intersection theory over schemes. We start by defining Chow rings and then deal with their functoriality. We then move on to vector bundles and how they interact with Chow rings, in particular looking at Chern classes. We conclude with a brief discussion of equivariant intersection theory.

The last chapters ***

%The complete list of the speakers and 
%\begin{enumerate}
%    \item Marino Domenico
%    \item Sorce Francesco
%    \item Leonardini Pietro
%    \item Gallese Andrea
%    \item Minnocci Francesco
%    \item ***
%\end{enumerate}


