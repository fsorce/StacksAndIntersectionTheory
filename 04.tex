\chapter{Intersection theory 1}
\begin{center}
	{\huge Speaker: Andrea Gallese}
\end{center}
\bigskip

\noindent
The main reference for this chapter is \cite{eisenbud20163264}.

\section{Motivation}

Think back to our roots: algebraic geometry is the study of solutions to systems of polynomial equations, which we can think of as intersecting some geometric objects.

In this chapter, we will consider schemes of finite type over algebraically closed fields. Eventually we will impose smoothness.

\begin{definition}[]
Let $X$ be a scheme as above. We define the \textbf{algebraic cycles} to be
\[Z(X)=\Z\spa{\text{integral closed subschemes of $X$}}=\bigoplus_{k}Z_k(X).\]
where $Z_k(X)$ is the same thing for subschemes of dimension $k$.
\end{definition}

\begin{remark}
$Z(X)=Z(X^{red})$.
\end{remark}

\begin{definition}
Given $Y\subseteq X$ closed subscheme, if $Y_1,\cdots, Y_r$ are its irreducible components the we define its \textbf{associated algebraic cycle} to be
\[\ps{Y}=\sum \ell_i[Y_i]\]
where\footnote{length of the module over itself} $\ell_i=\length(\Oc_{X,y_i})$ for $y_i\in Y_i$ the generic point.
\end{definition}


\begin{definition}[]
Let us consider the subgroup of $Z(X)$ given by
\[\Rat(X)=\ps{Z\cap (t_0\times X)-Z\cap (t_1\times X)}\]
where is some integral subvariety $Z\subseteq \Pj^1\times X$ and $t_0,t_1\in \Pj^1(k)$.

We define the \textbf{Chow group} as the quotient
\[CH(X)=\frac{Z(X)}{\Rat(X)}\]
\end{definition}

\begin{example}
$Z=\cpa{xy=t}\subseteq \Pj^1\times X$, and if we slice $Z$ at $t=0$ and $t=1$ we get $\cpa{xy=1}$ and $\cpa{xy=0}$, so these two give equivalent cycles.
\end{example}


\begin{remark}
The Chow group is graded by dimension and by codimension.
\end{remark}



\begin{example}
In $\A^2$ all points are rationally equivalent:

multiply by $\Pj^1$ and take the line connecting the two points
\end{example}

\begin{example}
We have an action $\GL_2\acts \A^2$ and $\GL_2\subseteq \A^4$. If $L$ is a line and $g\in \GL_2(k)$ then $[L]\sim [gL]$.

The idea is to take the line in $\A^4$ which connects $id\in \GL_2$ to $g\in \GL_2$.
\end{example}



Now that we have our objects that we want to intersect, we want to count the ``number of intersections" somehow.


\subsection{Trasversality}

\begin{definition}[]
Let $A,B$ be integral subvarieties of $X$. We say that $A$ and $B$ are \textbf{transverse} at $p\in A\cap B$ if $T_pA+T_pB=T_pX$, i.e. $\codim T_p A+\codim T_p B=\codim(T_pA\cap T_p B)$.


$A$ and $B$ are \textbf{generically transverse} if they are transverse at the generic points of the intersection.
\end{definition}


\begin{lemma}[Moving lemma]
The following propositions hold when $X$ is a smooth quasi-projective variety:
\begin{itemize}
\item For all $\al,\beta\in CH(X)$ there exist $A,B\in Z(X)$ such that $\al=[A]$, $\beta=[B]$ and $A,B$ are generically transverse
\item The class $[A\cap B]$ does not depend on the representatives $A,B$ from the previous point.
\end{itemize}
\end{lemma}

PICTURE OF TANGENT LINE MOVING TO A TRANSVERSE ONE

\begin{theorem}
Let $X$ be smooth and quasi-projective, then there exists a unique product on $CH(X)$ such that for all $A,B$ subvarieties of $X$, if they are generically transverse then $[A]\cdot [B]=[A\cap B]$.

This makes $CH(X)$ into a graded (by codimension), associative, commutative ring.
\end{theorem}

\begin{remark}
The identity of the Chow group of $X$ is the \textbf{fundemental class}, the class of the whole $[X]$.
\end{remark}


\begin{example}
Example of the two circles
\end{example}

\begin{theorem}
If $X$ is irreducible and $\dim X=n$ then $CH_n(X)\cong \Z$ and it is generated as a group by $[X]$.

If $X$ has irreducible components given by $X_1,\cdots, X_r$, then $[X_1],\cdots, [X_r]$ generate a free abelian subgroup of $CH(X)$ which is isomorphic to $\Z^r$.
\end{theorem}
\begin{proof}[Sketch]
If $Z\subseteq X\times \Pj^1$ gives some rational equivalence then $Z\subseteq X_i\times \Pj^1$ by irreducibility. So we may only consider the irreducible case.

If $Z\cap (t_0\times X)=X_i$ then $Z=X\times \Pj^1$ by irreduciblity.
\end{proof}

\section{Divisors}
\begin{proposition}
Suppose $X$ is irreducible of dimension $n$, then
\[CH_{n-1}(X)\cong \Pic(X)\]
the correspondence is that the divisor $D=\sum n_i Y_i$ corresponds to $\sum n_i[Y_i]$.
\end{proposition}
\begin{proof}[Proof (Well defined)]
If $f\in k(X)$ then we may take $Z=\cpa{f(x)=t}$ to see that $[\cpa{f=0}]=[\cpa{f=\infty}]$, so $\mathrm{div}(f)$ does map to $0$.

Also, $\Rat(X)$ is generated by the relations induced by principals divisors in this way.
\end{proof}



\begin{proposition}
The Chow ring of affine space is $CH(\A^n)=\Z[\A^n]\cong \Z$.
\end{proposition}
\begin{proof}
Let $Y$ be any positive codimension cycle not passing through $0$ (just move it a little if necessary).
We define
\[Z=\cpa{(t,tz)\mid z\in Y}=V(\cpa{f(z/t)\mid f\in I(Y)})\]
such an $f$ exists because $Y$ does not go through $0$ and so $f(z/\infty)=f(0)$, thus $Z$ defines an equivalence between $Y$ for $t=1$ and $\emptyset$ for $t=\infty$.
\end{proof}

\section{Mayer-Vietoris and Excision}

\begin{theorem}[Mayer-Vietoris]
Let $X_1,X_2\subseteq X$ closed subvarieties, then we have an exact sequence
\[CH(X_1\cap X_2)\to CH(X_1)\oplus CH(X_2)\to CH(X_1\cup X_2)\to 0.\]
\end{theorem}


\begin{theorem}[Excision]
Let $Y\subseteq X$ be a closed subvariety, then we have an exact sequence
\[CH(Y)\to CH(X)\to CH(X\bs Y)\to 0\]
\end{theorem}

\begin{corollary}
If $U\subseteq \A^n$ open, $CH(U)\cong \Z$ as an abelian group and it is generated by $[U]$.
\end{corollary}


\section{Functoriality / proper push-forward}

\begin{remark}
If $f:X\to Y$ is proper and $A\subseteq X$ subvariety then $f(A)^{red}\subseteq Y$ is a subvariety.
\end{remark}



\begin{theorem}
If $f:X\to Y$ proper, we can get $f_\ast:CH(X)\to CH(Y)$. On generators it behaves as follows
\[f_\ast[A]=\begin{cases}
0 &\text{if }\dim f(A)<\dim A\\
d[f(A)] &\text{if }\dim f(A)=\dim A\text{ and }d=\deg(A\to f(A))
\end{cases}\]
\end{theorem}


\begin{theorem}
$f_\ast$ defines a dimension-preserving morphism of the Chow-groups
\end{theorem}

PICTURE OF $A+B+C$ being equivalent to $D+E+F$, equivalent to $G+H$ MAPS TO STUFF WORKING OUT



\begin{proposition}
If $X\to \Spec k$ is proper, then we have a unique map $\deg:CH(X)=CH(pt)\cong \Z$ which preserves dimension (so $[Y]\mapsto 0$ if $\dim Y>0$ and $[pt]\mapsto 1$).
\end{proposition}



\section{Projective space}




\begin{theorem}
We have that
\[CH(\Pj^n)\cong \frac{\Z[t]}{(t^{n+1})}\]
where $t$ is the class of some linear hyperplane. Moreover, if $X\subseteq \Pj^n$ irreducible of codimension $k$ and degree\footnote{Hilbert polynomial nonesense} $d$ then $[X]=dt^k$.
\end{theorem}
\begin{proof}
\begin{lemma}[Affine stratification]
Consider the chain of inclusions
\[\cpa{pt}=\Pj^0\subseteq \Pj^1\subseteq\cdots\subseteq \Pj^n\]
and consider $U_i=\Pj^i\bs \Pj^{i-1}$ (affine, we are choosing an affine chart).

It follows that $CH(\Pj^n)$ is generated by the classes of $[\ol{U_i}]$.
\end{lemma}
\begin{proof}[Sketch]
Induction on numeber of strata + excision
\[Z=CH(\ol{U_0})=CH(U_0)\to CH(\Pj^n)\to CH(\Pj^n\bs \cpa{pt})\]
and the last bit is generated by $[\ol{U_i}]$ by inductive hypothesis.
\end{proof}


\begin{lemma}
Intersecting $k$ hyperplanes gives a $(n-k)$-plane, so $t^k=[(n-k)\text{-plane}]$.
\end{lemma}

\begin{lemma}
Let $M\subseteq \Pj^n$ be a $k$-plane, we have a surjective\footnote{$(n-k)$-plane $\cap$ $k$-plane $=[pt]$, generator of $CH^n(\Pj^n)$} map
\[CH^k(\Pj^n)\xrightarrow{\cdot[M]}CH^n(\Pj^n)\cong \Z\]
and the full composition is the degree of $-\cdot[M]$ (DRAW DIAGRAM LATER)
\end{lemma}
\end{proof}

\begin{exercise}
In $\A^2$ draw three circles $C_1,\ C_2,\ C_3$. How many circles can I draw which are tangent to $C_1$, $C_2$, $C_3$? Turns out it's 8.
\end{exercise}
\begin{proof}[Solution]
A circle in the plane is $(x-A)^2+(y-B)^2=C$. If we homogenize we get two points at infinity $[1:\pm i:0]$. We define a circle in $\Pj^2$ to be a conic which passes through these two points.

Recall that conics in $\Pj^2$ are parametrized by $\Pj^5$ by looking at the coefficients. By imposing the fact that the circles should go through two distinct points we get that circles are parametrized by a $\Pj^3$.


What does it mean for a circle to be tangent to $C$ fixed circle? Let $Z_C=\cpa{\text{circles tangent to $C$}}\subseteq\Pj^3$. Let
\[\Phi=\cpa{(r,D)\mid r\in C,\ D\in \text{circle tangent to $C$ at $r$}}\]
We have two projections $\Phi\to C$ and $\Phi\to Z_D$. Note that the fibers of $\Phi\to C$ are $\Pj^1$ (homotety of the same circle getting bigger and bigger). So $\Phi$ should be a surface of some kind. It should have degree 2 because tangency is checked by imposing that the discriminant of the quadratic equation obtained by substituting the equations of the two circles into each other is a quadratic.

Moreover, $Z_D$ is birational to $\Phi$ (from a circle in $Z_D$ we can get the point of tangency it has with $C$, giving the equivalence).

So $Z_{C_1}\subseteq\Pj^3$ is a surface of degree 2. In the Chow group $CH(\Pj^3)$ we have $2t=[Z_{C_1}]$ and so
\[[Z_{C_1}\cap Z_{C_2}\cap Z_{C_3}]=[Z_{C_1}][Z_{C_2}][Z_{C_3}]=(2t)^3=8t^3\]
and $8t^3\in CH(\Pj^3)$ is the class of eight points.
\end{proof}
















